


========== Symbols and nomenclature =====

    T       Temperature
    E       Energy
    S       Entropy
    H       Enthalpy
    G       Gibb's Free energy

    Z, z    Partition function (zustands-summe)
    Q       Reaction quotient (not partition function in this document)
    K       Equilibrium constant

    P, p    Probability
    k       Rate constant
    v       Reaction rate

    X or ξ  Extend of reaction
    Y       Yield of reaction (not always X, if there is by-products)


Constants:

    NA      Avogadro's constant = 6.022e23 / mol
    R, k    Gas constant and Boltzman's constant


=========== Generally ============

## Reaction: ##

                        v_hyb
    strandA + strandB   ------>  duplex
                       <------
                        v_melt

## Reaction quotient: ##

    Q_r = [duplex] / ([strandA] [strandB])


## Energy: ##

    ΔG = ΔH - TΔS           at standard conditions: ΔG° = ΔH° - T*ΔS°

    ΔG = ΔG° + RT ln(Q)     at standard conditions: ΔG = ΔG°, because Q = 1


## Partition functions: ##

    z_s = exp(-E_s/RT)                  # General partition function for state s
    => z_melted = exp(-E_melted/RT)     # Partition for the melted state
    => z_duplex = exp(-E_duplex/RT)     # Partition for the duplex state

    Z = sum(P_s for all states s)        # Partition function of the system

    When comparing just the two states, melted and hybridized duplex, we usually set
    E_melted = 0, giving us E_duplex = ΔE_hyb, where ΔE_hyb is the change from melted to duplex state.
    Since we have constant pressure and temperature, we can use Gibbs free energy as E: ΔE_hyb = ΔG_hyb
        z_melted = exp(-0/RT) = 1
        z_duplex = exp(-ΔG_hyb/RT)


## State probabilities: ##

    p_s = exp(-E_s/RT) / Z              # The general form:
     => p_melted = 1 / Z                # We have defined E_melted = 0
     => p_duplex = exp(-ΔG_hyb/RT) / Z  # Considering two isolated strands, probability of the two strands being hybridized.
     => p_duplex = exp(-ΔG_hyb/RT) / (1 + exp(-ΔG_hyb/RT))
                = 1/(1+exp(ΔG_hyb/RT))  # The minus in front of ΔG_hyb is lost during conversion.
    Since sum(p_s for all states s) = 1:
        p_melted = 1 - p_duplex

    In terms of ΔG° and reaction quotient Q:
        p_duplex = 1 / (1 + exp(ΔG°/RT)*Q)      # Because exp(ΔG/RT)*Q = exp((ΔG°+RT ln(Q))/RT) = exp(ΔG°/RT)*exp(ln(Q))


Kinetics

    v_hyb = k_hyb [strandA] [strandB]
    v_mel = k_mel [duplex]



======= At equilibrium: ==============


Equilibrium:

    K = Q   (definition of equilibrium constant K)

    K = [duplex] / ([strandA] [strandB])


If K = exp(-ΔG°/RT) is given:
    [strandA] = [strandA]_init - [duplex], [strandB] = [strandB]_init - [duplex]

    K   = [duplex]/([strandA]_init - [duplex])*([strandB]_init - [duplex])
        = D / (Ai - D)*(Bi - D) = D / (Ai*Bi - Bi*D - Ai*D + D^2)
    D/K = D^2 + (-Bi-Ai)*D + Ai*Bi
     0  = D^2 + (-Bi-Ai-1/K)*D + Ai*Bi

    Second-degree equation:
        0 = ax^2 + bx + c,      x=D, a=1, b = (-Bi-Ai-1/K), c = Ai*Bi
        det = sqrt(b^2-4ac)
        x = (-b ± sqrt(det) / 2a

    Insert known values for Ai, Bi and K, then:
        [duplex] = x
        [strandA] = [strandA]_init - [duplex]
        [strandB] = [strandB]_init - [duplex]



Thermodynamics at equilibrium:

    ΔG = 0
    => ΔG° = -RT ln(Q),     , and Q = K at T=Tm
    => K = exp(-ΔG°/RT)

    Relationship between temperature, enthalpy and entropy:
    ΔG° = ΔH° - T*ΔS°
        = -RT ln(K)         # Because

Kinetics and equilibrium:


    v_hyb = v_mel            # The amount formed equals the amount lost per time

    =>  k_hyb [strandA] [strandB] = k_mel [duplex]          (only true at equilibrium)

        k_hyb / k_mel = [duplex] / ([strandA] [strandB])
        k_hyb / k_mel = K




======= At T = Tm ==================

At T=Tm for strandA:

    [strandA] = [duplex] = [strandA]_init/2
    [strandB] = [strandB]_init - [duplex] = [strandB]_init - [strandA]_init/2

    Q   = [duplex] / ([strandA] [strandB])
        = [strandA]_init/2 / ([strandA]_init/2 * ([strandB]_init - [strandA]_init/2))
    1/Q = ([strandA]_init/2 * ([strandB]_init - [strandA]_init/2)) / [strandA]_init/2
        = [strandB]_init - [strandA]_init/2             # dnac2 = [strandA]_init/2, dnac1 = [strandB]_init
    Q   = 1 / ([strandB]_init - [strandA]_init/2)




======= Approximations for countable to uncountable (thermodynamic limit) =====


Stirling's approxmation:
     lnN! ≈ NlnN – N    (Actually lnN! = N ln(N) - N + ln(N)/2 + 1/ln(3) is essentially precise.)



======= Challenges =================


## Determine the relationship between k_hyb and k_mel ##

    At equilibrium, when v_hyb = v_melt:
        k_hyb = K * k_mel

    As always, at equilibrium, ΔG = 0 => Q = K = exp(-ΔG°/RT).
    So, k_hyb = exp(-ΔG°/RT) * k_mel...
    Or, k_mel = exp(+ΔG°/RT) * k_hyb

    Rates:
        v_hyb = k_hyb * [domain] * [compl]_total
        v_mel = k_mel * [duplex]
              = k_mel/2 * [duplexed domain or complement]           # since for every duplex we have two domains.
              = k_mel * [duplexed domain or complement] * p(0.5)    # Or flip a coin if you select a duplexed domain.

    Currently, all "concentrations" are considered during domain selection.
    After selection, we have
        p_hyb = self.hybridization_probability(domain1, domain2, T)
    which is calculated using binary_state_probability_cal_per_mol(deltaG, T, Q=1)
        p_hyb = p_i = 1 / (1 + math.exp(+ΔG°/(R*T))*Q)      # exp(-ΔE/kT) / (exp(-ΔE/kT) + 1) = 1/(1+exp(ΔE/kT))

    I am using p_melt = 1-p_hyb.

    We basically need to guarantee that, when a domain and it's partner has been selected, then:
        p_hyb = exp(-ΔG°/RT) * p_mel
    or
        p_mel = exp(+ΔG°/RT) * p_hyb

    Right now,
        p_hyb = exp(-ΔG°/RT)/(exp(-ΔG°/RT) + 1) = 1/(1 + exp(+ΔG°/(R*T)))
    and
        p_mel = 1 - p_hyb
              = 1 - exp(-ΔG°/RT)/(exp(-ΔG°/RT) + 1) = (exp(-ΔG°/RT) + 1 - exp(-ΔG°/RT))/(exp(-ΔG°/RT) + 1)
              = 1 / (exp(-ΔG°/RT) + 1)

    Is p_mel == exp(+ΔG°/RT) * p_hyb ?
        p_mel = exp(+ΔG°/RT) * p_hyb ,  insert p_hyb = exp(-ΔG°/RT)/(exp(-ΔG°/RT) + 1)
              = exp(+ΔG°/RT) * exp(-ΔG°/RT)/(exp(-ΔG°/RT) + 1)
              = 1 / (exp(-ΔG°/RT) + 1)
              = p_mel above
    ...it would seem so!
    However, it also seem that it is easier to simply define:
        p_hyb = 0.001   # Arbitrarily defined factor
        p_mel = exp(+ΔG°/RT) * p_hyb

    ...but then, it really isn't a probability, it is a rate factor:
        r_hyb = 0.1                         # Rate, given that a duplex and its partner has been selected.
        r_mel = exp(+ΔG°/RT) * 0.1          # Although ΔG° is usually very negative, so exp(+ΔG°/RT) is close to 0.

    Or, we could do the opposite:
        r_mel = 0.001                       # Then k_mel is independent on ΔG° (and T) -- seems wrong.
        r_hyb = exp(-ΔG°/RT) * 0.001        # But negative ΔG° means exp(-ΔG°/RT) can be very large (>> 1). Bad.

    But then again, now we suddenly have to ask:
    How do we decide whether to hybridize or de-hybridize based on r_mel?
    -- It only works if ΔG° < 0.

    UPDATE: Yes, using
        k_hyb = <a constant value, e.g. 1e5>, and
        k_mel = exp(+ΔG°/RT) * k_hyb
    is indeed the right approach.
    Strictly speaking, k_hyb varies with ionic strength and temperature, but that
    doesn't seem to be very significant.
    You can make it better if you account for whether there are any strong early
    interactions, e.g. GC pairs vs no Gc pairs, but that is not a major factor.


    ### OLD ###

    Currently, when using "domain selection":

        p_hyb   = p_selection * (p_hyb|selected)
                = [domainA_j]  * p_duplex(Q=1)

        For [domainA_j] = 1 uM,
            p_hyb = 1e-6 * 1 / (1 + exp(ΔG°/RT)*Q)

    No, wait. I have two rounds of selection, one for [domain]_i and the other for [compl]_total:
        v_hyb = k_hyb * [domain] * [compl]_total

        p_hyb = ? * p(select domain i) * p(finding a complementary domain)
        selecting _a_ domain has p=1, but we are looking at a specific domain. Whethe we consider one domain
        specie or all copies, we should still get the same.



## More kinetic considerations: ##

Relationship between rate constants:

        K = k_hyb / k_mel  <=> k_hyb = K * k_mel  <=> k_mel = (1/K) * k_hyb

    With K = exp(-ΔG°/RT):

        k_hyb = exp(-ΔG°/RT) * k_mel   <=>   k_mel = exp(+ΔG°/RT) * k_hyb

ᵗ‡

## Arrhenius equation, transition energy: ##

For negative ΔG° (hybridization), i.e. G°(hyb) < G°(melted)

    Transition state for hybridization:
        E‡(m->h) = E(melted) + ΔE‡ = E(hybridized) + ΔE(h->m) + ΔE‡
    I believe ΔE‡ is the same as the "initiation energy".
    ΔE‡ is positive, E(melted) = 0, and ΔE(h->m) = -ΔE(hyb)
    For melting:
        ΔE‡(melt) = ΔE‡ -ΔE(hyb)    # ΔE(hyb) < 0


What effect does this have on kinetic rate constants?

    Rate of hybridization (from melted to hybridized state):
        k(hyb) = A(hyb) * exp(-ΔE‡/RT)
    Rate of melting (from hybridized to melted state):
        k(mel) = A(mel) * exp((ΔE(hyb)-ΔE‡)/RT)
               = A(mel) * exp(-ΔE‡/RT) * exp(ΔE(hyb)/RT)

    Which is consistent with k_mel = k_hyb * exp(+ΔG°/RT),
    if and only if A(hyb) = A(mel) = A(T)   # A(T), since A is probably temperature dependent.

    We can just combine A and exp(-ΔE‡/RT) to a single factor:
        B(T) = A(T) * exp(-ΔE‡/RT)
    I belive this is just the same as k(hyb, T).


So, two duplexes with same ΔG° / Tm but different sequences will have the same K,
but different k_hyb and k_melt.


Temperature dependence:

    Ouldridge, Louis (oxDNA, NAR 2013):
""" Bimolecular association rate constants of 10^6 - 10^7 1/M/s have been
    measured at approximately room temperature and at high salt concentrations
    ([Na+] of 1 M or [Mg2+] 0.01 M) for DNA (7–9) and RNA (10–13).
    There is agreement that dissociation rates increase exponentially
    with temperature (8,10–12,14), but authors have reported association
    rates that increase (8,12), decrease (10,11) and behave non-monotonically
    (14) with temperature."""


Note: Wang and Moerner (Nat. Methods, 2014) has a nice table with litterature values
for hybridization and melting (k_on/k_off).
    It seems k_on is generally around 1e5 and 1e6 /s.
    k_on is strongly dependent on ionic strength and whether the sequence has any G-C basepairs to form a strong initial interaction.
    k_off depends on temperature and hybridization strength (ΔG°).

Thus, it is probably reasonable to set k_on to a fixed value, e.g.
    k_on = 1e5 # unit of /M/s
and let
    k_off = k_on/K = k_on*exp(+ΔG°/RT)          [k_on/k_off = K = exp(-ΔG°/RT)]
          = 1e5 * exp(+ΔG°/RT)  # unit of /s


Uh, note that http://www4.ncsu.edu/~franzen/public_html/CH433/lecture/Kinetics_Second_Order.pdf
states something quite different from this:
    "For strand lengths of 50 to 5000 nucleotides, the rate is
    directly proportional to the square root of the length of the
    shorter of the two strands For DNA/DNA association the shorter of the two strands. For DNA/DNA association, the
    maximum rate occurs 25 ° C below the Tm with only minimal
    variation in rate from 15 to 35 ° C below the Tm.
    The rate of association approaches zero near the Tm." <---- That last part, is that really right?
    -- No, can't be. If k_on is 0, then  (for say [s1]=[s2]=[duplex] = 1 uM)
        v_on  = k_on  [s1] [s2]  = 0 * 1e-6 * 1e-6 = 0
        v_off = k_off [duplex]   = 0.1 * 1e-6 = 1e-7        # 0.1 just used as example.
       but at T=Tm, v_on must equal v_off.
    Also, Tm is concentration dependent.
    It doesn't make sense that k_on would be concentration dependent.
    The hybridization rate (v_on), yes, but not the rate constant it self.  # v_on  = k_on  [s1] [s2]

    Although at very high temperatures, it might seem reasonable that k_on it self goes down:
    As we discussed with Arrhenius kinetics, ΔG° can impact the rate constants:
    At the usual regime where ΔG° < 0:

        k_on  = A exp(-ΔE‡/RT)
        k_off = A exp((+ΔG°-ΔE‡)/RT) = A exp(-ΔE‡/RT) exp(+ΔG°/RT)      # Remember, ΔG° < 0
              = k_on * exp(+ΔG°/RT)

    However, when T > Tm at molar conc, ΔG° > 0,
    then ΔG° is favoring the melted state, and hybridization is "up hill":

        k_on  = A' exp(-ΔE‡'/RT) exp(-ΔG°/RT)                           # Remember, ΔG° is now positive
        k_off = A' exp(-ΔE‡'/RT)

    Note that I put a prime on A' and ΔE‡' since they may not be the same as for ΔG°.
    (Both A and ΔE‡ are very likely dependent on temperature, salt, etc, so they are not exactly constant for ΔG° < 0)

    So yes, it is true: when we are above Tm *for molar concentrations* (which is *a lot*
    above Tm for typical uM concentrations) and ΔG° turns positive, then
    we will certainly start to k_on approaching zero exponentially.
    But I don't see that it should do that for T=Tm at typical working uM concentrations.



If we wanted k_on to have dependence on K or ΔG° at molar concentrations,
we could re-use my initial partition-function probability formulation:

    k_off = k_on/K, K = exp(-ΔG°/RT)

    k_on  = k_0 / (1 + 1/K)     # For K above 1, this is about k0, but for K < 1, this decreases fast.

    k_off = k_0 / (1 + K)

    At negative ΔG°, k0 is sort of a place-holder for A exp(-ΔE‡/RT).
    At ΔG°=0, K=1, so k_on = k0/(1+1/K) = k0/2.
    This is a bit different from my expectation that k_on = A exp(-ΔE‡/RT) would still hold.
    Not sure if the rate should be lower at ΔG°=0 (meaning the ΔE‡ is higher) ?
    But since it would certainly destabilise important intermediates, it makes sense.
    Actually, remembering the Ouldridge oxDNA paper, the intermediates is indeed important.
    Destabilizing these would act to reduce k_on at T < Tm for molar concentrations.
    Note again that both A and ΔE‡ depends on temperature, salt, etc.
    Also remember that the Arrhenius kinetic reaction rate and energy model is mostly suited for
    fast reactions with a single transition intermediate.
    For DNA hybridization, meta-stable intermediates are important enough that Arrhenius kinetics
    doesn't always fit.


As we approach T=Tm from below, we would expect k_off to increase.
We could have something like I previously had where
        p_on  = 1 / (1 + exp(ΔG°/RT))      # Satisfies p_on = p_off * exp(-ΔG°/RT) --reeely?
        p_off = 1 - p_on
Except there is no reason that k_on + k_off should equal some constant.

But, you could have:
    k_on  = k_0 / (1 + 1/K)
    k_off = k_on / K
          = k_0 / (1 + 1/K) / K = k_0 / ((1 + 1/K)*K) = k_0 / (K + 1)

    k_on/k_off  = k_0 / (1 + 1/K) / (k_0 / (K + 1))
                = 1 / ((1 + 1/K) / (K + 1)) = 1 / (((K+1)/(K+1) + (1/K)(K+1)/(K+1))/(K+1))
                = 1 / ( 1/(K+1) + (1/K)/(K+1) )
                = 1 / ( (1 + (1/K)) / (K+1) )   # Multiply both with K
                = 1 / ( (K + 1) / ((K+1)*K))    # Re-arrange
                = 1 / ( ((K + 1) / (K+1)) / K)    # Re-arrange
                = 1 / ( 1 / K )    # Re-arrange
                = K

Often at uM conc, at T=Tm, K will be around 1e6 /M and Kd will be 1 uM.
    [duplex] / ([s1] [s2]) = 1 uM / (1 uM * 1 uM) = 1 / 1 uM = 1e6 /M
As T goes up above Tm, K will decrease, going to 1 /M for Kd = 1 M, the temperature where T=Tm at 1 M concentrations.

For
    K = 1e6: 1/(1+1/K) =~ 1e6
    K = 1e6: 1/(1+1/K) =~ 1e6



## Kinetics data from litterature: ##

Gao Y, Wolf LK, Georgiadis RM, NAR 2006:
     Association DNA hybridization rate constants at
     20 degC in 0.5 M NaCl/TE, (1 uM of each strand):
     For unstructured strands (CTCTGAACGGTAGCATCTTGACAAC):
        1.0 M NaCl  0.5 M NaCl  0.1 M NaCl      Unit of
        16e5 /M/s   12e5 /M/s   4e5 /M/s
     Strands with some intrinsic secondary structure: (AGATCAGTGCGTCTGTACTAGCAGT)
        8e5 /M/s    7e5 /M/s    1e5 /M/s
     Strands with high intrinsic secondary structure: (AGATCAGTGCGTCTGTACTAGCACA)
        4e5 /M/s    2e5 /M/s    0.8e5 /M/s  (Nucleation, fast regime)
         - However, multiply by 4 if we assume nucleation of 6 bp instead of all 24.
        2e5 /M/s    0.5e5 /M/s  0.08e5 /M/s (Zippering, slow regime)
    Association rates for surfaces are 10-40 times slower.


Hybridization rate between two domains, both present at 0.1 uM:

    hybridization_rate = r_hyb = 4e5 /M/s * [domain1] * [domain2] =
        = 4e5 /M/s * 0.1e-6 M * 0.1e-6 M = 4e9 M/s   [duplex formation rate]

You can also say that you are just looking at a single, selected domain,
and you are considering the hybridization rate
against a single other domain, present in 0.1 uM:

    r_hyb = 4e5 /M/s * [domain2] =
        = 4e5 /M/s * 0.1e-6 M = 0.04 /s   [duplex formation rate for a fixed domain]

    This is a half-life of ln(2)/(0.04 /s) = 17 s.
    So, on average you have to wait 17 seconds before domain2 hybridizes.

Compare this to the (straightforward) melting rate: (depends heavily on dG_hyb!)

For a 10-mer, typically:

    r_mel = 3 /s

    This is a half-life of ln(2)/(3 /s) = 0.23 s.
    So, on average, domain2 stays attached for 0.23 s.


How to interpret: See next section.



## Kinetics and half-life ##

For first order reaction: duplex -> s1 + s2, half-life = ln(2)/k.
For second order formation: s1 + s2 -> duplex, with [s1] = [s2],
half-life of s1, s2 is 1/k/[s1 at t=0].
So, states that have first-order k = 10^4/s have half-life = 0.69/1e4 = 0.69e-4.
For k=1e6-1e7, half-life is 0.7e-6 - 0.7e-7.


## DNA hybridization energy calculation: ##


Algorithms:
    Nearest-neighbor - e.g. the DNA_NN4 table in biopython.
    NuPack
    UNAfold - http://unafold.rna.albany.edu
        DINAMelt http://unafold.rna.albany.edu/?q=DINAMelt/Hybrid2




## Converting to direct-time simulations (discrete SSA) ##


Gillespie:

Notes:
 * N chemical species S_1, …, S_N,
 * which interact through M chemical reactions R_1, …,R_M
 * Constant volume Ω, thermal equilibrium, well stirred, etc
 * X_i(t) denote the number of molecules of species S_i in the system at time t.
 * State-change vector ν_j = (ν_1j, ..., ν_Nj), ν_ij is the change in the S_i molecular population caused by one Rj reaction
     (If the system is in state x and one R_j reaction occurs, the system immediately jumps to state x + ν_j)
 * propensity function a_j for reaction R_j defined as:
    a_j(x)dx = probability that one R_j reaction will occur
    in the next infinitesimal time interval [t, t+dt] inside the volume Ω.
    Typically propensity function for R_j being the unimolecular reaction of species S_1 -> products:
        a_j(x) = c_j*x_1
    For bi-molecular reaction R_j between species S_1 + S2 -> products:
        a_j(x) = c_j*x_1*x_2
    For homo-reaction R_j: S_1 + S_1 -> products:
        a_j(x) = c_j*x_1*(x_1 - 1)/2
    Where x_1 is the number of specie S_1 molecules and x_2 is number of S_2 molecules.
    c_j is similar to k_on/k_off, except that for bi-molecular reactions it is
    inversely proportional to the volume Ω:
        c_j = k_j rate constant for uni-molecular reaction S1 -> (products)
        c_j = k_j/Ω for bi-molecular reactions S1+S2, except:
        c_j = 2*k_j/Ω for homo bi-molecular reactions with the same specie S1+S1.

Other hints:
 * Subscript "0" is used to indicate "all" or "total", i.e.
        a0(x) = sum(a_j for all j in 1..J))
    but it can also mean "at time t=0", e.g. ^x_0
 * Propensity function: the function whose product with dt gives the probability that a particular reaction will occur in the next infinitesimal time dt

Direct method algorithm:

    0. Initialize the time t = t₀ and the system's state x̄ = x̄₀.

    1. With the system in state x̄ at time t, evaluate all the aj(x̄) and their sum a0(x̄).

    2. Generate values for τ and j using Equations 10a,b (or their equivalent).
        a. Draw two random numbers, r1 and r2, in the interval ]0;1]
        b. Let τ = 1/a₀(x̄) ln(1/r1) = ln(1/r1)/a₀(x̄)
        c. Let J = the smallest integer satisfying sum(aj(x̄) for j in 1..J) > r2*a₀(x̄)

    3. Effect the next reaction by replacing t ← t + τ and x̄ ← x̄ + νj.

    4. Record (x̄, t) as desired. Return to Step 1, or else end the simulation.




## Example: ##

System with complementary strands: 1 s1 and 1 s2 in volume Ω = 1e-18 L (c = N / Ω / N_A = 1/1e-18/6.022e23 = 1.66 uM)
    Possible species are:
        i=1 S₁ = strand1 = st1
        i=2 S₂ = strand2 = st2
        i=3 S₃ = duplex12= d12
    - Initially unhybridized.
    Reactions and state-change vectors ν_j
        j=1   st1 + st2 -> d12,    ν = (-1, -1, +1)
        j=2   d12 -> st1 + st2,    ν = (+1, +1, -1)
    Reaction rate constants:
    - k_1 = k_on = 1e5 /M/s,
       c₁ = k_1/Ω = 1e5 /M/s /1e-18 = 1e23 /M/s = 0.16 L/s      # since 1 M = mol/L = 6.022e23/L
    - k_2 = k_off = 0.01 /s,
       c₂ = k_2 = 0.01 /s
    -  K  = k_on/k_off = 1e7 /M
    -  Kd = 1/K = k_off/k_on = 0.1 uM.  # We are above K_d, so should be mostly hybridized.
    Propencity functions:
        a₁(x̄) = c₁*x₁*x₂
        a₂(x̄) = c₂*x₃
 * Possible products are: s1, s2 and d12,
    Initial state vector: x̄₀ = (1, 1, 0)

 1. Evaluate all the aj (x) and their sum a0 (x).
    Propensity functions:
        a₁(x̄) = c₁*x₁*x₂ = 0.16*1*1 = 0.16     [for reaction j=1]  # unit of L/s
        a₂(x̄) = c₂*x₃ = 0.01*0 = 0             [for reaction j=2]  # unit of /s
    Sum: a₀(x̄) = 0.16 + 0 = 0.16.

 2. Generate values for τ and j:
    Draw two random numbers in the interval ]0;1]: r1 = 0.3, r2 = 0.7
        τ = 1/a0(x) ln(1/r1) = 1/0.16 ln(1/0.3) = 7.52  # seconds
    Let J = the smallest integer satisfying
        sum(aj(x̄) for j in 1..J) > r2*a0(x̄) = 0.7 * 0.16 = 0.112
    For J = 1, we have
        ∑ aj(x̄) = a₁(x̄) = 0.16, which is larger than r2*a0(x̄) = 0.112.

 3. Effect the next reaction by replacing t ← t + τ and x̄ ← x̄ + νj:
        t = t + τ = 0 + 7.52 = 7.52
        x = x̄ + νj = (1, 1, 0) + (-1, -1, +1) = (0, 0, 1)

Loop to 1:
 1. Evaluate all the aj(x̄) and their sum a₀(x̄).
    Propensity functions:
        a₁(x̄) = c₁*x₁*x₂ = 0.16*0*0 = 0    [for reaction j=1]  # unit of L/s
        a₂(x̄) = c₂*x₃ = 0.01*1 = 0.01      [for reaction j=2]  # unit of /s
    Sum:
        a₀(x̄) = 0.01

 2. Generate values for τ and j:
    Draw two random numbers in the interval ]0;1]:
        r1 = 0.91, r2 = 0.43                [high r1 -> reaction within a short time]
        τ = ln(1/r1)/a₀(x̄) = ln(1/0.91) / 0.01 = 9.43 s
    Let J = the smallest integer satisfying
        sum(aj(x̄) for j in 1..J) > r2*a0(x̄) = 0.43 * 0.01 = 0.0043
    For J = 2, we have
        ∑ aj(x̄) = a₁(x̄) + a₂(x̄) = 0 + 0.01, which is larger than r2*a0(x̄) = 0.0043

 3. Effect the next reaction by replacing t ← t + τ and x̄ ← x̄ + νj:
        t = t + τ = 7.52 + 9.43 = 16.95 s
        x = x̄ + νj = (1, 1, 0) + (+1, +1, -1) = (1, 1, 0)




₀₁₂₃


Implementation questions:
    I am going to have a lot of different uniquely different molecular species S,
    each with its own set of reactions. In fact, I cannot enumerate them from the onset,
    but will have to generate them dynamically during the simulation.
    This is all the different domains and their complex state.

    For each domain1 and its complement domain2, I have at least 5 different reaction conditions:
        domain1 is free, domain2 is free
        domain1 is free, domain2 in a complex (with state L2)
        domain1 in a complex (with state L1), domain 2 is free
        domain1 in a complex (with state L1), domain2 in a complex (state L2)
        domain1 in a complex (with state L1), domain2 IN THE SAME complex (also in state L1)

    So really we just have to make a list for each domain with the states:
    The rate constants are then combinations of these domain states:
                 -- domain1 --
                 1   2   3   4
         do  1  1.0 1.0 1.0 1.0
         ma  2  1.0 1.0 1.0 0.8
         in  3  0.8 0.7 0.5 0.3
         2   4  0.2 0.1 0.0 0.0

    In most cases, the reactivity will be the same as for the free domain.

    However, I'm not going to have all conditions at every step.
    I can still just go over my domains, classify them by "domain state specie" (domain in different conformation),
    but I only have to include the sub-species that I actually have. I.e.
    I don't have to include historical sub-species.
    Instead, I will keep an updated list of actual sub-species:
        Whenever a
    I keep a cache of historical sub-species and their reactions and rate constants, indexed as:
        [T][{(domain1-specie, complex-state), (domain2-specie, complex-state)}]
    e.g.
        [330][{()}]
    It is important that when I specify "domain1" + "complex state" that I can determine it uniquely.
    For instance, we might have a complex with multiple "domain1", each with different reactivities.
    I might have to include an identifier (domain1, complex-state, complex-domain-id) ?

    Anyway, at the start of each cycle, I expect that I have an up to date list of
        * domain sub-species, listed as
            (domain1, complex-state, count x)  -- for free domains, complex-state = 0.
          or maybe just a dict mapping 'domain_subspecies':
            (domain1, cstate) => list of domain1 molecules with this conformation.
        * Possible reactions R_j's, propencity constants c_j's and state change vectors, v_j's:
            ({(domain1, cstate), (domain2, cstate)}, c_j, HYB_TUPLE)
            # or it could be a dict keyed by:
                {(domain1, cstate), (domain2, cstate)} => c_j, is_hybridizing
            # Instead of a full state change vector, I use a dynamically interpreted bit noting whether
            # we hybridize or melt:
            # HYB_TUPLE = (-1, -1, +1) if hybridizing, MEL_TUPLE = (+1, +1, -1) if melting. Or just True vs False bit.

    1. Evaluate all propencity functions aj(x̄) and their sum a₀(x̄):

    # Edit: Do this AT THE END, UPDATE propencity functions aj(x̄) in-place after updating x̄.
    # Don't actually go over all domain_subspecies. Unless maybe to check that everything is OK.

        a = [c_j * prod(len(domain_subspecies(ds)) for ds in doms_spec)
             for doms_spec, c_j, is_hybridizing in possible_reactions_lst]  # doms_spec is the (domain-specie, cstate) above.

        a_sum = sum(a)

    2. Generate values for τ and j:  - easy.
        a. Draw two random numbers, r1 and r2, in the interval ]0;1]
        b. Let τ = 1/a₀(x̄) ln(1/r1) = ln(1/r1)/a₀(x̄)
            tau = ln(1/r1)/a_sum
        c. Let J = the smallest integer satisfying sum(aj(x̄) for j in 1..J) > r2*a₀(x̄)
            breaking_point = r2*a_sum
            j = 1
            sum_j = 0
            while sum_j < breaking_point:
                j += 1
                sum_j += a[j-1]   # python 0-based index: a[0] = j_1
            # We now have the j the smallest integer satisfying sum(aj(x̄) for j in 1..J) > r2*a₀(x̄)

    3. Effect the next reaction by replacing t ← t + τ and x̄ ← x̄ + νj.

        t = t + tau
        dom_spec, c_j, is_hybridizing = possible_reactions_lst[j]

        # Need to select an actual domain1+domain2 that will hybridize:
        dom1, dom2 = get_domains_by_domspec(dom_spec)
        if is_hybridizing:
            hybridize(dom1, dom2)
        else:
            dehybridize(dom1, dom2)

    3b. Update domain_subspecies dict, possible_reactions_lst and perhaps reaction_propencity_lst:

        domain_subspecies[dom_spec]


    4. Record (x̄, t) as desired. Return to Step 1, or else end the simulation.



    To bin or not to bin...
    Should I try to organize domains into sub-species according to their conformation?
    Yes. I need that anyways in order to efficiently store propencity constants for later re-use.


## Relationship between an ensemble of strands and one or two isolated (selected) strands ##


    1. Partition function


    2. Hybridization probability


    3. Reaction rate:

        Rate = k * prod(Ci^yi for i = 1..N)            # 5_60_lecture30.pdf
        where:  k  = rate constant, Ci = Concentration of reactant "i",
                yi = order of reaction with respect to reactant "i".





## What is the problem with the oversampling factor?  ("fudge factor")

    Consider the edge case where you only have two strands, each at 1 mM concentration
    and an oversampling factor of 1000.
    We are at T slightly above Tm, but below Tm for "standard conditions".
    1. Starting at the de-hybridized state,
    the probability of picking either of the two domains is 1 (duh),
    and the probability of then picking the other domain is c/M * of = 1.
    So, we have our (domain1, domain2) pair selected.
    The p_hyb is ~1, so we will usually switch to hybridized state.
    2. Next round, we calculate the probability of being in hybridized state.
    This probability will be high, so the probability of de-hybridizing will be low.
    But, when we multiply by the oversampling factor, the de-hybridization probability will
    be sufficiently high that we may switch state.
    This process continues for all steps in the simulation. Every time we record the state,
    there is approx 50 percent chance of being in either state.





======= References and litterature ==============


http://ocw.mit.edu/courses/chemistry/5-60-thermodynamics-kinetics-spring-2008/lecture-notes/




======= Other sotware packages =================


Thermodynamics
* NuPack


Kinetics, nucleic acid specific
* Multistrand

General-purpose thermodynamics and kinetics packages:
* MLAB - http://www.civilized.com/ (Good paper on the kinetics of "F + G -> B" system here: http://www.civilized.com/files/sobnew.pdf)
* Cantera - https://github.com/Cantera/cantera   [C++, Python, Matlab]
* ChemicalKinetics.jl - https://github.com/scidom/ChemicalKinetics.jl  [Julia]



